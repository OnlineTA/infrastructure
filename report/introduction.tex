\chapter{Introduction}
In this report, we will describe the design and implementation of an
infrastructure for a system facilitating submission and automated
correction of student assignments.

Unfortunately, some of the learning objectives proved overly ambitious
for the time frame that we had available. In particular, we had hoped
that our proof of concept implementation would be in a much more
complete than it currently is. This impairs our ability to present
firm conclusions but we do, however, expect that the design and
testing methodologies described provides a solid starting point for
such conclusions to be drawn at a later point.

A Master's project carried out in the spring of 2014 has laid the
foundation for this work by discussing the intricate aspects of
performing automated assessment of assignments\cite{onlineta}. This
work builds upon the previous project by describing an infrastructure
for a complete system supporting the submission and automated
assessment of assignments \fxnote{Rewrite last part}

\section{Background and motivation}
As a result of the study progress reform recently passed by the Danish
parliament, students are required to finish their university studies
faster than previously\cite{reform}. A consequence of this, is that many students
can no longer find the time to act as TA's alongside their studies.

This has sparked a general interest at DIKU\footnote{Department of
  Computer Science, University of Copenhagen} to investigate
implementing an automated assignment assessment system in order to
reduce the workload for teachers and teaching assistants TA's.

Since the assignments of many Computer Science courses require
students to complete programming homework, Computer Science finds
itself in a unique position with regard to take advantage of automated assignment
assessment systems.

The system that we here will describe does not intend to replace TA's
entirely, but rather automating the tedious of compiling an executing
student programs. Human involvement is still needed to asses
submissions as a whole.

An additional goal of this systeom is to be used as an Intelligent
Tutoring System (ITS)\cite{Freedman:2000:LIT:350752.350756} which can
provide support for students while they are solving their assignments.

%\section{Terminology}
%We shall describe some of the terminology that we will use henceforth
%in the report. This section is intended to clarify the
%contexts-specific interpretation of terms from which ambiguities could
%otherwise arise. \fxnote{Is this section even needed?}

%\begin{description}
%  \item[Assignment] Is given to students by course responsibles and
%    contains what you would exect.
%  \item[Evaluator] An evaluator is
%\end{description}

\section{Related work}
Tools for performing automated assessment of programming assignments
is currently an actively researched topic. As a result of this, a
large number of systems for performing this task has emerged over the
years \cite{Ihantola:2010:RRS:1930464.1930480}.




%%% Local Variables:
%%% mode: latex
%%% TeX-master: "master"
%%% TeX-command-extra-options: "-enable-write18"
%%% End:
