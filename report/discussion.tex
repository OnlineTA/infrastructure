\chapter{Conclusions}
The purpose of this project, was to serve as a starting point for the
design and implementation of the OnlienTA system. Thus, in the initial
phases of this project, a significant amount of time was needed simply
to figure out where we wanted to take the project and state
the requirements of the design.

This means, as previously stated, that we have not been able to
present a complete working implementation of our proposed
design. While this lack of definite results weakens our ability to
make firm recommendations, our proof of concept implementation is
sufficiently complete to enable us to confirm many of the
assertions that we have made.

In spite of this, that the design that we have proposed in this report
will provide a solid foundation for the continued work of implementing
the OnlineTA.




\section{Future work}
In the immediate term, the primary priority in the further development
of this product is to complete the implementation of the design
described in this report. This work is already well underway, but some
final pieces are missing before a complete and functioning system is
available.

Once our functioning system is available, we begin the process of
benchmarking our implementation in order to make recommendations on
the scale of the hardware required to support the load of assessing
the submissions of students.

Furthermore, a significant effort is also required by teachers in
order to write scripts for performing assessments

Looking ahead, a continuous effort for evaluating, maintaining and
improving the OnlineTA system in order to enable it to adapt to
changing demands of future courses.




%%% Local Variables:
%%% mode: latex
%%% TeX-master: "master"
%%% TeX-command-extra-options: "-enable-write18"
%%% End:
