\chapter{Measurements}
A goal of this project is to determine the servber capacity required by
the OnlineTA system. Since the current implementation is not yet at a
state where we can reliably perform such measurements, we will instead
focus om describing measurements which can be performed of the final
implementation in order to assess server capacity requirements.

%Furthermore, as the design that we have proposed offers seamless
%horizontal scaling, we

\section{Metrics of submission assessment}
In order to know what benchmarks to conduct we need to establish the
performance metrics that we can use to estimate the expected
performance of our system.


%\begin{description}
%\item[Network bandwidth requirements]
%\item[Disk space requirements]
%\item[Disk IO]
%\item[Memory requirements]
%\item[CPU performance]
%\end{description}

\section{Performance thresholds}
In order to prevent making overestimated hosting recommendations it's
important to establish acceptable thresholds for assessment
throughput. A system that is scaled toward delivering assessments "`as
fast as possible"' under any circumstance will have significantly
different requirements than a system where we accept queuing of
assignments and thus delaying assessments during peak load.

The perceived usefulness of the system and overall student
satisfaction will depend greatly on striking the right balance between
acceptable assessment throughput thresholds and cost of hardware
deployments.

\section{System scaling requirements}
As our initial scope is limited to performing assessment of
the assignments of DIKU coursess, we can develop an
idea of how large our assignment throughput needs to be fairly easily. E.g if we
have 100 students enrolled in the introductory programming course, we
can calculate an upper bound on the amount of resources
consumed. This can be done by calculating the expected resource
requirements for each assignment in the course. Our required
throughput is then 

\section{Estimating static requirements}
We define static requirements as resources that scales linearily wuth
an assignment, e.g. if we need to estimate the requried


\section{Benchmarks}
In order to perform a complete evaluation of the system, we need to
perform a th



%%% Local Variables:
%%% mode: latex
%%% TeX-master: "master"
%%% TeX-command-extra-options: "-enable-write18"
%%% End:
